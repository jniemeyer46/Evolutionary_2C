\documentclass[•]{article}
\usepackage{graphicx}
\usepackage{listings}
\graphicspath{ {/Users/jjniemeyer46/Desktop/Pics/EC_2b} }

\usepackage{color}
 
\definecolor{codegreen}{rgb}{0,0.6,0}
\definecolor{codegray}{rgb}{0.5,0.5,0.5}
\definecolor{codepurple}{rgb}{0.58,0,0.82}
\definecolor{backcolour}{rgb}{0.95,0.95,0.92}
 
\lstdefinestyle{mystyle}{
    backgroundcolor=\color{backcolour},   
    commentstyle=\color{codegreen},
    keywordstyle=\color{magenta},
    numberstyle=\tiny\color{codegray},
    stringstyle=\color{codepurple},
    basicstyle=\footnotesize,
    breakatwhitespace=false,         
    breaklines=true,                 
    captionpos=b,                    
    keepspaces=true,                 
    numbers=left,                    
    numbersep=5pt,                  
    showspaces=false,                
    showstringspaces=false,
    showtabs=false,                  
    tabsize=2
}
\lstset{style=mystyle}

\author{John Niemeyer\\JJNB78@mst.edu}
\title{COMP SCI 5401 FS2017 Assignment 2c}

\begin{document}
\maketitle

\section*{\begin{center}MOEA Explained\end{center}}

For this experiment I decided to go with a fairly simple approach for my CoEA.  So I have a configurable variable called CoevolutionaryFitnessSamplePercent that can be set to any percent a user likes, this variable should be in the form of \% (so 100 would be what it would be set at for 100\%), it will get converted into a percentage during the execution of the program automatically.  Once I had that variable all that was left was to get the number of opponents each bracket would have.  In order to do that I simply make a variable that was set to the formula $(\mu + \lambda - 1) * CoevolutionaryFitnessSamplePercent$ in order to obtain the number of opponents for that experiment.1

\section*{\begin{center} Experiment parameters and graphs \end{center}}

\section{IPD Results}

\subsection{Graphs}
\noindent \includegraphics [scale=0.65] {/graph}

\pagebreak
\subsection{Result Tables}
Problem 1a: final results\\\\
\noindent \includegraphics [scale=0.65] {/results}

\pagebreak
\subsection{Statistical Analysis}
\noindent \includegraphics [scale=0.65] {/statistical_analysis}\\\\
\indent So according to the statistical analysis (shown above) the p-value for the best fitness is not low enough to say that the results are statistically significant.  That means that the t-value of -0.18387, computed using the tables given, were not far enough apart from the t-value given of 2.045 to make the difference in the fitness values statistically significant.

\pagebreak
\subsection{EA Configurations}
If you want to get the same results you have to change the newSeed variable to 0 (Zero) in the configuration file in order to use the previous seed.\\\\

Using config1.txt 
\begin{lstlisting}
runs = 30
fitness = 10000

k = 5
d = 10
l = 30
n = 5
mu = 0.01
lambda = 2
parentNumber = 5
p = 1
terminationEvals = 3


prob_log_file = logs/log1.txt
prob_solution_file = solutions/solution1.txt


Initialize: Ramped_halfandhalf = 1

parentSelection: Fitness_Proportional_Selection = 1, Over_Selection = 0

Recombination: subTree_Crossover_Recombination = 1

Mutation: subTree_Crossover_Mutation = 1

survivalSelection: Truncation = 1, kTournament = 0

bloatControl: parsimonyPressure = 1

Termination: numEvals = 1, noChange = 0

newSeed = 1
\end{lstlisting}

\end{document}